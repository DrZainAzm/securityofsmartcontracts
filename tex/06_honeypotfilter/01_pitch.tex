Although the idea of honeypots has been documented in various blog posts, still not every smart contract developer is aware of this phenomenon. Because of this, up until time of writing, honeypot contracts have been published on a regular basis -- and still users were tricked by honeypots into sending Ether to that address, as for example recently seen at the contract address \cite{etherscan:latchedhoneypot}.

Although some of them get recognized by users of the blockchain explorers, posting comments and therefore warning potential victims, many honeypots remain undetected. Despite Etherscan having implemented a warning for honeypots\footnote{that was used for example in \cite{etherscan:uncalledcallhoneypot}}, this feature is not used on a regular basis.

If a discovery note is posted at the page of a honeypot address, the owner usually closes the honeypot quickly, since potential victims would be scared away by the comments, making this honeypot a loss for the owner.

This is especially important because there are deployment gas costs for each honeypot of around 0.004 Ether -- which makes honeypots profitable if one in 200 created honeypots is successful when omitting the time spent for programming, deployment and setup. When looking at the transactions of a common honeypot pattern, the honeypot creator requires around ten minutes to initialize the honeypot; including time for code modifications and the verification on Etherscan the total deployment takes probably around 15 minutes.\footnote{ \((2.53 + 0.23 + 0.19 + 0.98 + 0.12 + 0.12) \texttt{ Szabo} = 4.17 \texttt{ Szabo} = 0.00417 \texttt{ Ether} \) as gas costs, \( 1 \texttt{ Ether} \) as the potential profit, 10 minutes between 11:57 and 12:07 on 16th of July 2018 as the time consumed by deployment, as taken from \cite{etherchain:honeypot}}

Because of this, reliable recognition of honeypots would stop honeypots from being profitable and therefore discourage their creators from uploading them to the blockchain. But since manual analysis can not be performed instantaneously after the contract has been uploaded to the blockchain and is time-consuming, a automated tool to recognize smart contract honeypots would help out here.

As a mitigation to the honeypot problem the author of this thesis has developed the tool \textsc{Amphicyon}\footnote{The name \textsc{Amphicyon} was taken from the bear-dog that populated North America, Europe, Asia and Africa for about 14 million years until it went extinct three millions years ago. The naming of the tool after that mixture of a bear and a dog is meant to emphasize the ability of the tool, as a "detection dog" and a "honeypot-seeking" bear.}, which is meant to be an automated honeypot filter for blockchain explorers. \textsc{Amphicyon} uses static code analysis of the Solidity code to find programming patterns used in honeypots, and decides for a given contract if it has the traits of a honeypot.

The project is written in typescript and has a size of more than 4000 SLoC and features 18 classifiers, that are able to detect all underhanded Solidity patterns presented in section \ref{section:honeypot:underhanded} along with other honeypot coding patterns.
