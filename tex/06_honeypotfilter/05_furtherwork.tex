\subsection{Ideas for Improvements}
To protect users from being tricked by honeypots, \textsc{Amphicyon} could be integrated into blockchain explorers like \url{etherscan.io} to automatically scan new verified contracts before they are published to the "Verified Contracts"-list. This can be easily done since no information that is not available on Etherscan is used. At the entry of detected honeypots then a warning above the contract information could be displayed, as already done for \cite{etherscan:uncalledcallhoneypot}; and the contract could be removed from the list of verified smart contracts. Except for honeypots, most contracts don't rely on being found by chance in the "Verified Contracts"-List of the blockchain explorer; so the damage of false positives would be limited. Since \textsc{Amphicyon} also marks the affected parts of the code for each honeypot technique, the finding could possibly also be displayed inside the source-code viewer at the corresponding line.

Creating honeypots which are optimized not to be recognized by \textsc{Amphicyon} would be difficult in that situation, because already known honeypots could not be deployed automatically any more without being detected. Once \textsc{Amphicyon} is programmed to detect a technique, their inventors would have to spend time to find new honeypot techniques. Additionally, the effort to create effective honeypots would be increased, since it becomes more difficult to write harmless-looking and deceptive code, if fewer options are available.

To improve the detection of honeypots, there are also several possible approaches:
\textsc{Amphicyon} could start using patterns in the transaction data (including internal zero-value-transactions, that are not displayed on \url{etherscan.io}, but on other blockchain explorers like \url{etherchain.org}) to detect the setup of the honeypots; this could help to lower the false positive rate and also detect honeypots that static code analysis is not able to reveal.

Additional classifiers could also help to increase the precision of the detection: For example, a contract could be checked for comparisons between the incoming Ether in \mintinline{solidity}{msg.value} and a fixed value, which is often used to incentivise users to send an amount making the honeypot worth the effort; or checking for a comparison \mintinline{solidity}{msg.value == tx.origin}, which was often a protection mechanism used by honeypots against attacks from other contracts (as presented in section \ref{section:honeypots:transactionordering}).

% New honeypot type: https://etherscan.io/address/0x2ed2a3a09568c8d16f1f3d03eeb1d1ea96e4a4dd#code