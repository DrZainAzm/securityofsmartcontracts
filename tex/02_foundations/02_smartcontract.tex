\subsection{Smart Contracts}
A smart contract is an "autonomous agent" stored on the blockchain. In the context discussed in this thesis, they are treated equally to the accounts of corresponding private key pairs, but are controlled by its code that is stored on the blockchain and executed by the nodes. To enable more advanced applications, every contract has a persistent storage which is part of the blockchain state that is kept over multiple transactions. The execution of smart contracts can be seen as a more advanced state transition of the blockchain.

Smart contracts are created as a result of a special creation message that can be launched by any account -- so that both external agents and other smart contracts can create new smart contract accounts. Once deployed, their code becomes immutable.

The abstract concept of smart contracts was first formalized in a note by Nick Szabo back in 1997 (\cite{szabo:smartcontracts}) who compared his idea to vending machines: Every account present on the blockchain is able to interact with the smart contract, so that both users and other smart contracts can send messages to the contract. Additionally, smart contracts are self-enforcing and unambiguous, since their outcome is determined by their deployed code -- and once deployed they are completely independent from their creator and can not be revoked.

On the Ethereum blockchain, smart contracts for many different applications can be found: While one of the more classical ideas is to use them to trade with or create own crypto-currencies (as done for example in \cite{etherscan:earlytokensale}), there were also projects using smart contracts fully-automated investment-company (\cite{etherscan:dao}), prediction markets enabling people to bet on certain events (\cite{augurproject:whitepaper}) or even different kinds of games like a the crypto-collectible game CryptoKitten (\cite{cryptokitten:whitepaper}).
